\section{Implementação das correções às vulnerabilidades detetadas}

Nesta secção, descreve-se a arquitetura tecnológica do projeto e detalham-se as intervenções técnicas realizadas para mitigar as vulnerabilidades identificadas (secção \ref{sec:vulnerabilidades-detetadas}).

\subsection{Stack Tecnológica e Arquitetura da Solução}

O objeto de estudo é uma aplicação web, inicialmente concebida e gerada na plataforma de inteligência artificial Base44. O processo de preparação para os testes incluiu a geração de funcionalidades através de IA, a aquisição de direitos de acesso para exportação e a subsequente transferência do código-fonte do \textit{frontend} para um repositório GitHub. É importante salientar que a auditoria é realizada com acesso exclusivo ao código do lado do cliente (testes \textit{White Box} ao \textit{frontend}).

Em termos técnicos, o projeto foi implementado como uma \textit{Single Page Application} (SPA) na linguagem JavaScript, com recurso à biblioteca React e à ferramenta de construção Vite. A estrutura interna dos ficheiros do projeto pode ser dividida em dois grupos lógicos.

O primeiro grupo refere-se à configuração padrão da \textit{framework} (\textit{Scaffolding}). Estes ficheiros não são gerados pela rede neuronal de forma única para cada pedido, sendo, pelo contrário, ficheiros-tipo e imutáveis para a maioria dos projetos baseados nesta tecnologia. Neste grupo incluem-se ficheiros de configuração como o \texttt{vite.config.js} (definições do construtor), o \texttt{package.json} (lista de dependências), bem como os pontos de entrada da aplicação — \texttt{index.html} e \texttt{main.jsx}.

O segundo grupo engloba os componentes e a lógica gerados diretamente pela inteligência artificial da Base44. Estes ficheiros encontram-se, predominantemente, nos diretórios \texttt{src/components} e \texttt{src/pages}. São estes que contêm o código único da interface e a lógica de negócio da aplicação específica, constituindo o foco principal da análise.

Uma vez que não existe acesso à infraestrutura real do servidor (\textit{backend}), a arquitetura da aplicação foi adaptada para execução local. A interação com o \textit{backend} é emulada através de um SDK local, onde os pedidos reais à API foram substituídos por um serviço \textit{Mock}, que devolve dados de teste.

\subsection{Metodologia de Testes}

A análise de segurança foi realizada sobre uma aplicação web alojada na plataforma Base44, num contexto de acesso limitado à componente servidor. Por esse motivo, todos os testes foram conduzidos com abordagem \textit{black-box}, centrada no comportamento observado a partir do lado do cliente, bem como na análise dos recursos públicos disponibilizados pela aplicação. Nesta secção descrevem-se, de forma detalhada, as ferramentas utilizadas e a metodologia seguida.

\subsubsection{cURL: Inspecção de Comunicação HTTP, CORS e Respostas do Servidor}

A ferramenta cURL desempenhou um papel central no processo de testagem. Através dela foi possível examinar, de forma granular, o comportamento do servidor perante pedidos legítimos e alterados. A inspeção de cabeçalhos através de pedidos \texttt{HEAD} permitiu observar diretivas de segurança como \texttt{X-Frame-Options}, \texttt{Referrer-Policy} e configurações de \textit{cache}, enquanto pedidos com cabeçalhos manipulados possibilitaram avaliar a política CORS implementada pela aplicação. A utilização de diferentes valores de \texttt{Origin} tornou possível compreender até que ponto o servidor valida pedidos externos.

O cURL foi igualmente utilizado para aceder diretamente a recursos estáticos, nomeadamente ficheiros JavaScript, imagens e potenciais ficheiros \texttt{.map}. Testes adicionais permitiram compreender o comportamento do servidor em cenários de tentativas de SSRF, bem como as diferenças entre pedidos efetuados para o domínio e para o endereço IP, revelando assim a atuação do CDN (Cloudflare) em diferentes contextos.

\subsubsection{ffuf: Fuzzing de Caminhos e Descoberta de Estruturas Ocultas}

O ffuf foi aplicado para explorar a estrutura de caminhos da aplicação e identificar potenciais áreas ocultas. Embora o ambiente Base44 utilize um modelo típico de SPA, onde múltiplos caminhos devolvem a mesma página inicial, o ffuf permitiu detetar quando um caminho correspondia a um recurso real ou quando era apenas tratado pelo \textit{router} interno do \textit{frontend}. Este processo ajudou a compreender o alcance público da aplicação e permitiu identificar quaisquer \textit{endpoints} ou páginas menos visíveis, contribuindo para a cartografia geral da superfície de exposição.

\subsubsection{Gobuster: Enumeração Complementar de Recursos Estáticos}

O Gobuster foi utilizado como complemento ao ffuf, sobretudo focado na enumeração de ficheiros e ativos estáticos. Esta abordagem permitiu confirmar o grau de homogeneidade das respostas do servidor e identificar quaisquer padrões invulgares que pudessem sugerir a existência de recursos adicionais. Apesar de a estrutura da aplicação ser bastante uniforme, esta ferramenta reforçou a verificação da consistência do comportamento do servidor perante pedidos fora do padrão.

\subsubsection{sqlmap: Testes Negativos de SQL Injection}

O sqlmap foi mobilizado com o objetivo de realizar testes negativos de SQL Injection, avaliando se parâmetros acessíveis ao utilizador poderiam estar vulneráveis a manipulação. Embora nenhuma tentativa de injeção tenha tido sucesso e o firewall de aplicação (WAF) tenha bloqueado diversas variantes dos testes, a sua utilização permitiu observar o modo como a plataforma Base44 reage a \textit{payloads} anómalos. O sqlmap contribuiu assim para verificar a robustez da aplicação e para documentar que não foram identificados pontos suscetíveis a este tipo de ataque.

\subsubsection{grep: Análise Estática de Bundles JavaScript}

Uma parte significativa da análise centrou-se nos \textit{bundles} JavaScript descarregados do \textit{frontend}, utilizando grep como ferramenta principal de processamento textual. Através dele foi possível procurar ocorrências de potenciais segredos embutidos nos ficheiros, como chaves de API, \textit{tokens} ou \textit{endpoints} internos. A ferramenta permitiu igualmente identificar a presença de padrões associados a vulnerabilidades de XSS, por exemplo inserções diretas via \texttt{innerHTML} ou o uso de funções potencialmente perigosas como \texttt{eval}.

O grep serviu ainda para procurar referências a mapas de origem (\texttt{sourceMappingURL}), o que permitiu determinar se existiam informações de \textit{debug} expostas publicamente, e para detetar eventuais casos de conteúdo misto (HTTP dentro de HTTPS). Este tipo de análise revelou-se essencial para avaliar a higiene e segurança do código entregue ao navegador.

\subsubsection{Qualys SSL Labs: Avaliação da Configuração TLS}

Foi utilizada a plataforma Qualys SSL Labs para avaliar a configuração TLS da aplicação. Este teste permitiu obter uma visão completa das versões de TLS suportadas, dos conjuntos de cifragem utilizados e da presença de mecanismos como o \textit{Perfect Forward Secrecy}. A verificação incluiu também a pesquisa por vulnerabilidades amplamente conhecidas, como Heartbleed, ROBOT, DROWN, POODLE e SWEET32. A análise fornecida pelo SSL Labs permitiu validar que, do ponto de vista criptográfico, a aplicação se apoia numa configuração robusta providenciada pelo Cloudflare.

\subsection{Metodologia Aplicada}

A metodologia seguiu uma sequência estruturada de fases, combinando técnicas de análise passiva, inspeção manual e testes ativos. Cada etapa incluiu ferramentas específicas, selecionadas de acordo com o tipo de evidência procurada e com as limitações inerentes ao ambiente Base44.

\begin{enumerate}
    \item \textbf{Footprinting e recolha inicial de informação}\\
    Nesta fase procurou-se compreender o comportamento geral da aplicação e do ambiente CDN que a envolve. Foram analisados os cabeçalhos HTTP, o comportamento do servidor perante pedidos simples e a forma como o Cloudflare processa diferentes origens.\\
    \textbf{Ferramentas utilizadas:} cURL, Qualys SSL Labs.
    
    \item \textbf{Mapeamento da aplicação e descoberta de recursos}\\
    Seguiu-se a cartografia dos recursos expostos ao cliente. Este processo incluiu a identificação de \textit{bundles} JavaScript, ativos estáticos, diretórios acessíveis e o modo como o \textit{router} da aplicação responde a caminhos inexistentes.\\
    \textbf{Ferramentas utilizadas:} ffuf, Gobuster, cURL.
    
    \item \textbf{Análise estática do frontend}\\
    Após recolha de ficheiros JavaScript, procedeu-se a uma análise detalhada do seu conteúdo, com foco na deteção de potenciais riscos de XSS, exposição de chaves, referências internas e presença de \textit{source maps}.\\
    \textbf{Ferramentas utilizadas:} grep, cURL.
    
    \item \textbf{Testes ativos de segurança}\\
    Foram realizados testes práticos destinados a avaliar a reação da aplicação a entradas inesperadas, incluindo \textit{fuzzing} de caminhos, manipulação de cabeçalhos CORS, pedidos malformados e tentativas controladas de SQL Injection.\\
    \textbf{Ferramentas utilizadas:} ffuf, Gobuster, cURL, sqlmap.
    
    \item \textbf{Avaliação da camada TLS e configuração criptográfica}\\
    Procedeu-se à análise aprofundada da configuração TLS do servidor, avaliando protocolos, cifragens e mecanismos de segurança adicionais.\\
    \textbf{Ferramentas utilizadas:} Qualys SSL Labs.
\end{enumerate}

\subsection{Relatório de Vulnerabilidades}

A Tabela \ref{tab:vuln} descreve todas as vulnerabilidades para as quais a aplicação foi testada e os respectivos resultados.

\begin{table}[H]
    \caption{Relatório de Vulnerabilidades}
    \label{tab:vuln}
    \centering
    \begin{tabularx}{\textwidth}{@{} >{\bfseries}p{4.5cm} >{\RaggedRight}X p{3cm} @{}}
        \toprule
        Vulnerabilidade & Como foi testada & Detectada? \\
        \midrule
        
        1. SQL Injection (SQLi) & 
        Testes com sqlmap sobre parâmetros GET, incluindo diferentes níveis, riscos e delays. & 
        Não detectada \\
        \addlinespace
        
        2. CORS Misconfiguration & 
        Pedidos curl com Origin manipulada; comparação de comportamentos em IP e domínio. & 
        Não detectada \\
        \addlinespace
        
        3. Exposição de Source Maps & 
        Teste directo com curl para obter .js.map, pesquisa de sourceMappingURL nos bundles. & 
        Parcial \\
        \addlinespace
        
        4. Exposição de Segredos / API Keys & 
        grep sobre JS para tokens, chaves, endpoints internos. & 
        Não detectada \\
        \addlinespace
        
        5. Mixed Content & 
        grep buscou referências http:// nos JS bundles. & 
        Parcial \\
        \addlinespace
        
        6. DOM-based XSS & 
        Procura em JS (grep) por usos de innerHTML, eval, insertAdjacentHTML. & 
        Possível \\
        \addlinespace
        
        7. Open Redirect & 
        Pesquisa de padrões location.href = ... redirect/url em JS. & 
        Detectada \\
        \addlinespace
        
        8. TLS Vulnerabilidades & 
        Avaliação completa via SSL Labs (Heartbleed, SWEET32, ROBOT, etc.). & 
        Não detectada \\
        \addlinespace
        
        9. Weak Cipher Suites & 
        Results do SSL Labs (TLS 1.0 / 1.1). & 
        Não detectada \\
        \addlinespace
        
        10. Headers de Segurança Ausentes & 
        curl -I analisando X-Frame-Options, Referrer-Policy, etc. & 
        Maioria presente \\
        \addlinespace
        
        11. SSRF via Host Header & 
        Tentativas de usar a app como proxy com curl -x. & 
        Não detectada \\
        \addlinespace
        
        12. Directory Enumeration & 
        ffuf e Gobuster. & 
        Não detectada \\
        \addlinespace
        
        13. Insecure Third-Party Scripts & 
        Análise das referências nos JS bundles (Rewardful, tracking scripts). & 
        Não detectada \\
        \addlinespace
        
        14. Tab-nabbing & 
        Busca manual de \texttt{<a target="\_blank">} sem \texttt{rel="noopener"}. & 
        Detectada \\
        \addlinespace
        
        15. Client-side Cache & 
        Busca de service worker e caches.open. & 
        Não detectada \\
        
        \bottomrule
    \end{tabularx}
\end{table}

\subsection{Implementação das correções}

Abaixo descrevem-se as correções aplicadas para cada vulnerabilidade detetada.

\subsubsection{Desativação de source maps em produção}

A mitigação desta vulnerabilidade focou-se na alteração das configurações de construção (build) da aplicação para o ambiente de produção. Embora os source maps sejam ferramentas fundamentais durante a fase de desenvolvimento — permitindo rastrear erros até aos ficheiros originais — a sua presença em produção não acrescenta valor funcional ao utilizador final e expõe desnecessariamente a estrutura interna do código.

A disponibilização destes ficheiros permitiria a qualquer utilizador, através das ferramentas de desenvolvimento do navegador (DevTools), aceder ao código não minificado, incluindo nomes de componentes, lógica de funções e comentários deixados pelos programadores.

Para corrigir esta falha, a configuração do bundler Vite foi ajustada para impedir a geração destes ficheiros na versão final. A alteração foi realizada no ficheiro \texttt{vite.config.js}, definindo explicitamente a propriedade \texttt{sourcemap} como \texttt{false} no objeto de configuração de \texttt{build}.

A implementação técnica é apresentada abaixo:

\begin{verbatim}
// vite.config.js
export default defineConfig({
  // ... outras configurações
  build: {
    sourcemap: false, // Desativa a geração de source maps em produção
  },
});
\end{verbatim}

Com esta alteração, o processo de compilação (\texttt{npm run build}) gera apenas os ativos minificados essenciais, mantendo o código-fonte original oculto. Esta medida reduz a superfície de reconhecimento para potenciais atacantes, sem afetar a funcionalidade da aplicação ou o processo de desenvolvimento local, onde os mapas continuam disponíveis.
\subsubsection{Injeções Inseguras no DOM}

Durante a análise foi identificado no ficheiro \texttt{chart.jsx} o uso de 
\texttt{dangerouslySetInnerHTML}, uma função do React que permite a inserção direta 
de código HTML no DOM. Este mecanismo é reconhecido como um \emph{sink} inseguro, 
uma vez que qualquer valor não validado inserido no atributo \texttt{\_\_html} 
pode ser interpretado pelo navegador como código executável, permitindo a ocorrência 
de ataques do tipo DOM-based XSS. A Figura~\ref{fig:codigo-vulneravel-xss} apresenta o 
trecho vulnerável.

\begin{figure}[H]
    \centering
    \includegraphics[width=0.95\textwidth]{capitulos/07_parte_pratica_PASTA/7.2.2_Injecoes_inseguras_no_DOM/img/Figura2.png}
    \caption{Exemplo de código vulnerável XSS}
    \label{fig:codigo-vulneravel-xss}
\end{figure}

Para mitigar esta vulnerabilidade, a geração dinâmica de estilos foi reescrita de forma 
a eliminar completamente o uso de \texttt{dangerouslySetInnerHTML}. Em vez de injetar HTML 
diretamente no DOM, a nova abordagem constrói a folha de estilos como uma \emph{string} 
controlada e atribui-a ao elemento \texttt{<style>} através de \emph{children} seguros do 
React, impedindo a interpretação de conteúdo arbitrário. O código corrigido encontra-se 
representado na Figura~\ref{fig:codigo-corrigido-xss}.

\begin{figure}[H]
    \centering
    \includegraphics[width=0.95\textwidth]{capitulos/07_parte_pratica_PASTA/7.2.2_Injecoes_inseguras_no_DOM/img/Figura3.png}
    \caption{Exemplo do código corrigido}
    \label{fig:codigo-corrigido-xss}
\end{figure}

Para além da remoção do \emph{sink} vulnerável, foram ainda aplicadas validações adicionais,
tais como o uso de \texttt{config || \{\}} para prevenir acessos inesperados a propriedades 
indefinidas e filtragem de valores nulos antes da construção das regras CSS.

\subsubsection{Reforço da Content-Security-Policy (CSP)}

Durante a fase de implementação das medidas de mitigação, a equipa identificou que a aplicação não possuía qualquer política explícita de \emph{Content-Security-Policy} (CSP), o que aumentava significativamente o risco de execução de código malicioso no navegador, sobretudo no contexto das vulnerabilidades relacionadas com manipulação do DOM descritas em secções anteriores.

A CSP é um dos mecanismos mais eficazes para reduzir a superfície de ataque de XSS no \emph{frontend}, uma vez que impõe restrições rígidas sobre as origens permitidas para scripts, estilos, imagens e outros tipos de conteúdo. Para colmatar esta lacuna, foi introduzida uma \texttt{meta tag} CSP no ficheiro \texttt{index.html} da aplicação, com o objetivo de limitar todas as fontes externas não autorizadas e reforçar a segurança do cliente.

A política final implementada encontra-se abaixo:

\begin{figure}[H]
    \centering
    \includegraphics[width=0.95\textwidth]{capitulos/07_parte_pratica_PASTA/7.2.3_Reforco_da_CSP/img/correcao1.png}
    \caption{Politica final implementada de Content-Security-Policy}
    \label{fig:figuracorrecaoCSP}
\end{figure}

Esta política define regras claras sobre os conteúdos que a aplicação pode carregar. Os scripts e os estilos passam a vir apenas da própria aplicação ou de ligações HTTPS seguras. Os conteúdos incorporados através de \texttt{object-src} deixam de ser permitidos, evitando o uso de ficheiros ou plugins que possam representar um risco. As imagens ficam limitadas a origens consideradas seguras e controladas. Para além disso, todos os pedidos de rede passam a ser feitos através de ligações seguras, reduzindo a probabilidade de interferências externas.


\subsubsection{Correção das Ligações Externas Inseguras (Reverse Tabnabbing)}

No âmbito das correções de segurança implementadas, foi realizada uma procura direcionada aos componentes de \textit{frontend} da aplicação para identificar todas as instâncias de navegação externa insegura. Foram detetados elementos de âncora (\texttt{<a>}) que utilizavam a propriedade \texttt{target="\_blank"} sem as devidas medidas de isolamento de contexto.

Conforme evidenciado nas alterações de código (ver Figura \ref{fig:reverse_tabnabbing_codigo}), a correção foi aplicada nos ficheiros \texttt{src/pages/Layout.jsx} e \texttt{src/pages/Profile.jsx}. Nestes componentes, existiam \textit{links} direcionados para perfis externos (LinkedIn) que expunham a aplicação ao risco de \textit{Reverse Tabnabbing}.

\begin{figure}[hbtp]
    \centering
    \includegraphics[width=0.8\linewidth]{capitulos/07_parte_pratica_PASTA/reverse_tabnabbing_PASTA/img/reverse_tabnabbing_codigo.png}
    \caption{Mitigação de \textit{Reverse Tabnabbing}. \textit{Diff} de código evidenciando a correção de segurança aplicada nos componentes Layout.jsx e Profile.jsx através da adição dos atributos rel.}
    \label{fig:reverse_tabnabbing_codigo}
\end{figure}

A mitigação consistiu na injeção explícita do atributo \texttt{rel="noopener noreferrer"} nas tags afetadas. Esta alteração instrui o navegador a:

\begin{itemize}
    \item \textbf{Noopener}: Garantir que a nova aba é instanciada num processo ou contexto separado, definindo \texttt{window.opener} como nulo.
    \item \textbf{Noreferrer}: Impedir o envio do cabeçalho \textit{Referer}, assegurando que o site de destino não recebe informações sobre a origem do tráfego.
\end{itemize}

Após a recompilação e execução da aplicação, verificou-se que a abertura destes \textit{links} externos mantém a funcionalidade esperada (abertura em nova aba), mas elimina o vetor de ataque anteriormente existente.

\subsubsection{Correção da Vulnerabilidade de Open Redirect}

Durante a análise do fluxo de autenticação verificou-se que a aplicação utilizava a expressão:

\begin{verbatim}
User.loginWithRedirect(window.location.href)
\end{verbatim}

A utilização de \texttt{window.location.href} representa um risco, pois o URL completo pode ser manipulado pelo utilizador antes do processo de autenticação. Por exemplo, um atacante pode adicionar parâmetros como:

\begin{verbatim}
/add-today?redirect=https://malicious.com
\end{verbatim}

Caso o código confie no URL completo, existe a possibilidade de provocar um \textit{open redirect}, levando o utilizador a um domínio externo após o login.

Para eliminar este risco, o código foi modificado para usar apenas o caminho interno da aplicação:

\begin{verbatim}
User.loginWithRedirect(window.location.pathname)
\end{verbatim}

As alterações realizadas no código-fonte podem ser visualizadas na Figura \ref{fig:github_open_redirect}.

\begin{figure}[hbtp]
    \centering
    \includegraphics[width=0.95\textwidth]{capitulos/07_parte_pratica_PASTA/open_redirect/img/open_redirect_github.png}
    \caption{Diff do GitHub mostrando a correção do Open Redirect}
    \label{fig:github_open_redirect}
\end{figure}

O \texttt{pathname} contém apenas a rota interna e ignora completamente domínios externos, parâmetros e fragmentos adicionados manualmente. Assim, qualquer tentativa de manipulação externa desaparece no momento da construção do URL de redirecionamento. Desta forma, elimina-se o vetor de ataque e garante-se que o redirecionamento ocorre sempre apenas dentro da própria aplicação.

\subsection{Resultados Obtidos}

No levantamento inicial, foram identificadas 5 vulnerabilidades principais que comprometiam a confidencialidade e integridade da aplicação. Estas falhas permitiam desde a leitura do código-fonte original até à execução de scripts maliciosos no navegador da vítima (conforme detalhado na Secção \ref{sec:vulnerabilidades-detetadas}).
Após a aplicação das medidas de mitigação, realizou-se um segundo ciclo de testes. A Tabela \ref{tab:resumo-correcoes} resume o estado atual das vulnerabilidades:

\begin{table}[H]
    \caption{Resumos do estado das vulnerabilidades identificadas após as correções.}
    \label{tab:resumo-correcoes}
    \centering
    % Definição das colunas ajustada para o conteúdo desta tabela
    % Coluna 1: Negrito, 4cm
    % Coluna 2: Normal, 3.5cm (para caber "Não mitigada")
    % Coluna 3: X (preenche o resto), alinhado à esquerda
    \begin{tabularx}{\textwidth}{@{} >{\bfseries}p{4cm} p{3.5cm} >{\RaggedRight\arraybackslash}X @{}}
        \toprule
        Vulnerabilidade & Estado no 2.º Ciclo & Observação \\
        \midrule
        
        Source Map Exposure & Mitigada & 
        Ficheiros \texttt{.map} deixaram de ser carregados pelo navegador (Erro 404 confirmado). \\ 
        \addlinespace
        
        Mixed Content & Não mitigada &
        Não é possível corrigir no frontend: não existe qualquer recurso HTTP no código, portanto o mixed content vem do SDK Base44, que não pode ser modificado. \\ 
        \addlinespace
        
        DOM-based XSS & Mitigada &
        \textit{Payloads} de teste (\texttt{<script>alert(1)</script>}) são renderizados como texto simples e não executam. \\ 
        \addlinespace
        
        Open Redirects & Mitigada &
        Tentativas de redirecionamento para \texttt{evil.com} resultam no retorno seguro à Homepage. \\ 
        \addlinespace
        
        Reverse Tabnabbing & Mitigada &
        A tentativa de definir a \texttt{propriedade\_location} resulta numa falha. \\
        
        \bottomrule
    \end{tabularx}
\end{table}

De seguida apresentam-se os resultados comparativos entre o estado inicial da aplicação e o estado após a implementação das correções de segurança.



\paragraph{Validação das Correções}

\noindent\textbf{Cenário Vulnerável (Antes da Correção)}

Quando os source maps estão ativos, as ferramentas de desenvolvimento do navegador conseguem reconstruir e exibir a estrutura completa dos ficheiros de origem. Conforme visível na figura~\ref{fig:source-tabs-enabled}, é possível inspecionar nomes de componentes, a lógica interna das funções e até comentários deixados pelos programadores, revelando detalhes desnecessários em produção que facilitam a engenharia inversa.

\begin{figure}[H]
  \centering
  \includegraphics[width=0.6\textwidth]{capitulos/07_parte_pratica_PASTA/source_tabs_disabling_PASTA/img/source_tabs_enabled.png}
  \caption{Inspeção com source maps ativados — código de origem reconstruído (antes da correção).}
  \label{fig:source-tabs-enabled}
\end{figure}

\noindent\textbf{Cenário Mitigado (Após a Correção)}

Após a alteração da configuração para \texttt{sourcemap: false} e a recompilação do projeto, a inspeção aos recursos carregados mostra apenas uma lista reduzida de ativos agrupados (bundled assets). O código original deixa de estar exposto, sendo apresentado apenas na sua forma minificada e ofuscada. Esta medida oculta a lógica interna, dificultando significativamente a análise por parte de terceiros ou de software de prospeção (scraping). A figura~\ref{fig:source-tabs-disabled} mostra o resultado observado após a correção.

\begin{figure}[H]
  \centering
  \includegraphics[width=0.6\textwidth]{capitulos/07_parte_pratica_PASTA/source_tabs_disabling_PASTA/img/source_tabs_disabled.png}
  \caption{Inspeção com source maps desativados — ativos minificados/ofuscados (após a correção).}
  \label{fig:source-tabs-disabled}
\end{figure}

\noindent Em ambos os cenários, as capturas foram obtidas através das ferramentas de desenvolvimento do navegador (DevTools). A correção reduz o ruído de informação disponível em produção e aumenta a dificuldade de engenharia inversa, sem afetar o desenvolvimento local, onde os source maps podem continuar a ser gerados.

\subsubsection{Injeções Inseguras no DOM}

Após a aplicação das correções, repetiram-se os testes de injeção com \emph{payloads} típicos de XSS (\texttt{<img onerror=alert(1)>}, \texttt{"<script>alert(1)</script>"}), e verificou-se que o conteúdo passou a ser tratado exclusivamente como texto dentro das regras CSS, sem qualquer execução de código. Isto confirma que o risco de DOM-based XSS associado a este componente foi eliminado.

Procedeu-se à validação prática da mitigação da vulnerabilidade DOM-based XSS. O objetivo do teste foi verificar se um atacante conseguiria injetar código JavaScript através do parâmetro \texttt{config.color}, utilizado internamente na geração dinâmica de estilos do componente \texttt{ChartContainer}.

Para este teste, foi construída uma prova de conceito (PoC) onde o valor do parâmetro \texttt{color} era substituído por um \emph{payload} malicioso contendo uma tentativa explícita de injeção de \texttt{script}:

\begin{verbatim}
const payload = 'red; } <script>window.HACKED = true</script> /*'
\end{verbatim}

Caso o componente ainda estivesse vulnerável, o código contido dentro da tag \texttt{<script>} seria interpretado pelo navegador e o valor \texttt{window.HACKED} passaria a ser definido como \texttt{true}, permitindo detectar a execução de código arbitrário no contexto da aplicação.

O componente foi então renderizado com esta configuração maliciosa, e no ciclo de vida \texttt{useEffect()} foi registado no console o valor observado de \texttt{window.HACKED}, conforme o exemplo:

\begin{figure}[hbtp]
    \centering
    \includegraphics[width=0.75\textwidth]{capitulos/07_parte_pratica_PASTA/7.2.2_Injecoes_inseguras_no_DOM/img/print4.png}
    \caption{Output do console após a injeção do \emph{payload} malicioso}
    \label{fig:codigo-corrigido}
\end{figure}


Através do resultado, percebe-se que o \emph{script} injetado não foi interpretado pelo navegador. Isto porque o \emph{payload} incluía uma tag \texttt{<script>}, que num cenário vulnerável seria incorporada no DOM e executada. No entanto, após a correção, a string é incorporada apenas como conteúdo literal dentro de regras CSS, não sendo processada como HTML. Como consequência, o navegador não executa o código JavaScript malicioso, pelo que a variável global \texttt{window.HACKED} não é criada. O valor \texttt{undefined} confirma a ausência de execução de código, indicando que:

\begin{itemize}
    \item nenhuma variável global com o nome \texttt{HACKED} foi definida;
    \item nenhuma instrução JavaScript proveniente do utilizador foi executada;
    \item o conteúdo gerado no \texttt{<style>} é tratado como texto puro e não como HTML interpretável.
\end{itemize}

Assim, o teste comprova que o mecanismo de injeção anteriormente vulnerável foi neutralizado. A prova de conceito demonstra que, após remover o uso de \texttt{dangerouslySetInnerHTML} e substituir o mecanismo de geração dinâmica de CSS por um processo controlado e não interpretável como HTML, o componente deixa de permitir que valores fornecidos pelo utilizador originem execução de \emph{scripts} no navegador.

O resultado \texttt{undefined} constitui evidência experimental de que a tentativa de XSS falhou, validando a eficácia da correção e garantindo que o componente \texttt{ChartContainer} já não apresenta a vulnerabilidade identificada inicialmente.

\subsubsection{Validação da Eficácia da CSP}

Para validar a eficácia da \emph{Content-Security-Policy} (CSP), simulámos tentativas reais de injeção de scripts externos e carregamento de conteúdos potencialmente perigosos. Para isso, utilizámos o DevTools do navegador e executámos os seguintes testes:

\begin{enumerate}
    \item \textbf{Tentativa de injeção de um script externo:} 
    Este teste teve como objetivo verificar se a CSP integrada impede o carregamento de scripts provenientes de origens não autorizadas, uma técnica comum em ataques XSS. Para isso, foi criado dinamicamente, através da consola do navegador, um elemento \texttt{<script>} cujo \texttt{src} apontava para um domínio malicioso. 
    Ao tentar introduzir o script na página, o navegador bloqueou de imediato a ação e apresentou uma mensagem de violação da política, indicando que a diretiva \texttt{script-src} estava a ser aplicada corretamente. 
    Este comportamento confirma que a aplicação deixou de permitir a execução de scripts externos, reduzindo significativamente o risco de injeção de código malicioso.
    
    \begin{figure}[H]
        \centering
        \includegraphics[width=0.95\textwidth]{capitulos/07_parte_pratica_PASTA/7.2.3_Reforco_da_CSP/img/evidencia1.png}
        \caption{Violação da Content-Security-Policy}
        \label{fig:codigo-vulneravel}
    \end{figure}

    \item \textbf{Tentativa de carregar um objeto potencialmente inseguro:} 
    Neste teste procurou-se validar se a diretiva \texttt{object-src 'none'} da CSP estava a impedir o carregamento de conteúdos incorporados, como ficheiros \texttt{.swf} ou outros objetos externos historicamente associados a vulnerabilidades. 
    Utilizou-se um elemento \texttt{<object>} apontado para um ficheiro remoto suspeito, inserido diretamente na consola. 
    Tal como esperado, o navegador bloqueou o carregamento e emitiu um aviso explícito de que a ação violava a política de segurança definida. 
    Este resultado demonstra que a aplicação está protegida contra a execução de objetos inseguros, reforçando o controlo sobre conteúdos que possam representar um risco adicional.
    
    \begin{figure}[H]
        \centering
        \includegraphics[width=0.95\textwidth]{capitulos/07_parte_pratica_PASTA/7.2.3_Reforco_da_CSP/img/evidencia2.png}
        \caption{Executar código inline não autorizado}
        \label{fig:codigo-vulneravel}
    \end{figure}

    \begin{figure}[H]
        \centering
        \includegraphics[width=0.95\textwidth]{capitulos/07_parte_pratica_PASTA/7.2.3_Reforco_da_CSP/img/evidencia3.png}
        \caption{Resultado de código inline não autorizado}
        \label{fig:codigo-vulneravel}
    \end{figure}

    \item \textbf{Tentativa de executar código inline não autorizado:} 
    O objetivo deste teste foi confirmar que a política de segurança também impede a execução de código inline, como acontece com o uso de \texttt{eval()}, frequentemente explorado em cenários de XSS. 
    Ao tentar executar \texttt{eval("alert('XSS')")} na consola, o navegador recusou a ação e apresentou uma mensagem indicando que a política não permite a avaliação de strings como JavaScript. 
    O facto de o alerta não ter sido exibido demonstra que a CSP está a impedir corretamente a execução de código dinâmico não autorizado, garantindo maior robustez contra ataques que dependem da manipulação do DOM.
    
    \begin{figure}[H]
        \centering
        \includegraphics[width=0.95\textwidth]{capitulos/07_parte_pratica_PASTA/7.2.3_Reforco_da_CSP/img/evidencia4.png}
        \caption{Unsafe eval bloquedo pelo CSP}
        \label{fig:codigo-vulneravel}
    \end{figure}
\end{enumerate}

Podemos assim concluir que a integração da Content-Security-Policy reforçou de forma significativa a segurança do frontend, introduzindo uma camada de proteção que a aplicação não possuía. A política definida reduz a superfície de ataque ao controlar as origens permitidas para scripts, estilos, imagens e objetos incorporados, mitigando riscos associados a XSS e outras execuções de código não autorizadas no navegador. Os testes demonstraram que todas as tentativas de carregar conteúdo externo ou executar código inline foram bloqueadas, confirmando a eficácia da solução. Em conjunto, esta intervenção tornou o comportamento da aplicação mais seguro e alinhado com as boas práticas de proteção no lado do cliente.


\subsubsection{Correção das Ligações Externas Inseguras (Reverse Tabnabbing)}

Para demonstrar a mitigação da vulnerabilidade, foi executada uma Prova de Conceito (PoC) simulando um cenário de ataque, antes e depois da mitigação, onde a página externa força o redirecionamento da aplicação original para um site arbitrário (neste exemplo, \url{https://www.google.com}).

O teste consistiu na injeção do seguinte comando JavaScript na consola da nova aba aberta: \texttt{window.opener.location = 'https://www.google.com';}

\noindent\textbf{1. Cenário Vulnerável (Antes da Correção)}

Conforme ilustrado na Figura \ref{fig:reverse_tabnabbing_antes}, ao executar o comando na consola da aba de destino, a aba de origem (que continha a aplicação Base44) foi imediatamente redirecionada para o Google. Isto comprova que, num cenário real, um atacante poderia substituir a aplicação legítima por uma página de \textit{phishing} idêntica, levando o utilizador a inserir as suas credenciais num site falso.

\begin{figure}[H]
    \centering
    \includegraphics[width=0.8\linewidth]{capitulos/07_parte_pratica_PASTA/reverse_tabnabbing_PASTA/img/reverse_tabnabbing_antes.png}
    \caption{Demonstração da vulnerabilidade de Reverse Tabnabbing. A execução do comando na consola (direita) forçou a aba original (esquerda) a navegar para google.com sem o consentimento do utilizador.}
    \label{fig:reverse_tabnabbing_antes}
\end{figure}

\noindent\textbf{2. Cenário Mitigado (Após a Correção)}

Após a aplicação dos atributos \texttt{rel="noopener noreferrer"}, o mesmo teste foi repetido. Conforme evidenciado na Figura \ref{fig:reverse_tabnabbing_depois}, a tentativa de definir a propriedade \texttt{.location} resultou numa falha, uma vez que o objeto \texttt{window.opener} é agora nulo. A aba original permaneceu segura e inalterada na aplicação Base44.

\begin{figure}[H]
    \centering
    \includegraphics[width=0.8\linewidth]{capitulos/07_parte_pratica_PASTA/reverse_tabnabbing_PASTA/img/reverse_tabnabbing_depois.png}
    \caption{Bloqueio do ataque de Reverse Tabnabbing. A tentativa de redirecionamento gera um erro na consola (TypeError: Cannot set properties of null), confirmando que a aplicação original está protegida contra manipulação externa.}
    \label{fig:reverse_tabnabbing_depois}
\end{figure}

Nota sobre Navegadores Modernos
É importante notar que, embora navegadores modernos baseados em Chromium (v.\,88+) apliquem implicitamente o comportamento \textit{noopener} em links \texttt{target="\_blank"}, a correção explícita implementada é essencial por dois motivos:

\begin{itemize}
    \item Garante a compatibilidade e segurança em navegadores mais antigos (\textit{Legacy Support});
    \item Aplica a diretiva \textit{noreferrer}, que não é assumida por defeito pelos navegadores, protegendo assim a privacidade dos dados de navegação do utilizador.
\end{itemize}

\subsection{Validação da Correção em Ambiente de Desenvolvimento}

Para confirmar a eficácia da correção, foi utilizado um mock do sistema de autenticação Base44 em ambiente de desenvolvimento. O objetivo foi observar diretamente qual URL a aplicação envia para a função \texttt{loginWithRedirect()} após a alteração implementada.

Primeiro, configurou-se o mock para simular o estado ``não autenticado'' na primeira execução. Isto obriga a aplicação a invocar o redirecionamento, permitindo inspecionar o valor utilizado.

\begin{verbatim}
throw new Error("Mock unauthenticated");
\end{verbatim}

Após gerar o erro intencional, a aplicação chama:

\begin{verbatim}
loginWithRedirect(window.location.pathname)
\end{verbatim}

O mock regista no console o valor recebido e, em seguida, executa um redirecionamento real com esse valor. Isto torna possível confirmar se a aplicação utiliza realmente apenas caminhos internos.

Foram testados dois cenários:

\begin{itemize}
    \item acesso normal à aplicação (por exemplo, \texttt{/add-today});
    \item tentativa de ataque usando parâmetros maliciosos:
    \begin{verbatim}
/add-today?redirect=https://malicious.com
    \end{verbatim}
\end{itemize}

Em ambos os casos, o comportamento observado foi o mesmo: o mock registou sempre apenas o valor interno:

\begin{verbatim}
/add-today
\end{verbatim}

A evidência deste comportamento no console do navegador é apresentada na Figura \ref{fig:devtools_open_redirect}.

\begin{figure}[hbtp]
    \centering
    \includegraphics[width=0.95\textwidth]{capitulos/07_parte_pratica_PASTA/open_redirect/img/open_redirect_devtools.png}
    \caption{Validação da correção via Developer Tools}
    \label{fig:devtools_open_redirect}
\end{figure}

Os parâmetros externos, incluindo URLs maliciosas, não foram utilizados no processo de redirecionamento. Isto confirma que, após a correção, o valor completo do URL deixa de ser relevante e qualquer manipulação feita pelo utilizador é ignorada automaticamente pela lógica da aplicação.

Assim, conclui-se que a vulnerabilidade de \textit{open redirect} foi eliminada com sucesso.


