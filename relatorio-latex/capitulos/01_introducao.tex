\section{Introdução}

\subsection{Enquadramento e Propósito do Estudo}

O uso crescente de plataformas \textit{no-code}, como a Base44 (\url{https://base44.com}), veio simplificar o desenvolvimento de aplicações web, mas também trouxe novos desafios em termos de segurança \cite{ref54}. Estas soluções permitem que utilizadores construam aplicações funcionais sem qualquer contacto com o servidor ou com o código do \textit{backend}, o que reduz significativamente as possibilidades de análise e reforço da segurança.

Este estudo tem como objetivo analisar, avaliar e corrigir vulnerabilidades existentes ao nível do \textit{frontend} de uma aplicação desenvolvida na Base44. Procura-se demonstrar que, mesmo com acesso apenas ao código do lado do cliente (\textit{HTML}, \textit{CSS} e \textit{JavaScript}), é possível identificar riscos que podem comprometer a integridade, a confidencialidade e a disponibilidade (CIA) da aplicação.

O trabalho assume um carácter exploratório e prático, focando-se na reprodução de falhas reais e na implementação de soluções corretivas. O principal objetivo é contribuir para a consciencialização e para o reforço da segurança em ambientes \textit{no-code}, onde o controlo técnico por parte dos programadores é limitado, mas onde os riscos continuam a ser relevantes.

Todo o trabalho desenvolvido, incluindo este relatório em formato \LaTeX\;e os ficheiros do estudo prático, encontra-se disponível num repositório Git alojado no GitHub no seguinte endereço: \url{https://github.com/acavieira/SRSI}. O acesso ao repositório permite a consulta integral dos conteúdos, promovendo a transparência e facilitando a reprodução dos testes realizados.

\subsection{Estrutura do Trabalho}

Após esta introdução, o relatório está organizado de forma lógica e progressiva, acompanhando o desenvolvimento do projeto desde os seus fundamentos teóricos até aos testes após as correções implementadas para corrigir as vulnerabilidades encontradas.

A secção seguinte apresenta uma visão geral das principais abordagens de teste de segurança web, com a explicação dos modelos de \textit{pentesting} utilizados: \textit{caixa-preta}, \textit{caixa-branca} e \textit{caixa-cinzenta}. Nesta parte, é também justificada a escolha da abordagem de \textit{caixa-cinzenta}, por se adequar ao contexto específico da aplicação analisada.

De seguida, serão detalhadas as ferramentas utilizadas no processo de análise, bem como a justificação técnica para a sua seleção.

Posteriormente, são apresentadas as vulnerabilidades identificadas na aplicação, cada uma acompanhada de uma explicação teórica, exemplos práticos, respetivas implicações de segurança, correções efetuadas testes sobre as mesmas.

Numa fase final, o relatório aborda vulnerabilidades previstas que não chegaram a ser confirmadas durante os testes e termina com uma reflexão crítica sobre as limitações da plataforma Base44, relacionando os resultados obtidos com os desafios mais amplos da segurança em ambientes \textit{no-code} baseados em inteligência artificial.
