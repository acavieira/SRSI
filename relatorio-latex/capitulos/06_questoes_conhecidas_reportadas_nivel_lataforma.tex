\section{Questões Conhecidas e Reportadas ao Nível da Plataforma (Base44)}

Para além das vulnerabilidades específicas detetadas na aplicação em si, é também essencial considerar as limitações estruturais da plataforma de alojamento — neste caso, a Base44.

Tal como documentado em fóruns de utilizadores e fontes técnicas, a Base44 adota uma arquitetura no-code/low-code baseada em deployments edge através da infraestrutura da Cloudflare, o que traz benefícios de desempenho, mas também restrições de controlo a nível de servidor [16], [2].

Estas limitações têm implicações diretas na superfície de ataque, na gestão de cabeçalhos de segurança, e na rapidez de resposta a incidentes. A seguir detalham-se os principais pontos identificados.

\subsection{Infraestrutura Baseada em Cloudflare e Deployments Edge}

\subsubsection{Enquadramento Teórico}

A arquitetura edge consiste na distribuição de conteúdos através de múltiplos nós geograficamente dispersos (edge servers), com o objetivo de reduzir a latência e aumentar a disponibilidade [17].

Neste modelo, o controlo direto sobre o servidor de origem é frequentemente substituído por políticas de cache e regras de proxy geridas pela rede CDN (Content Delivery Network).

Embora este modelo proporcione resiliência, escalabilidade e mitigação de DDoS, também implica que o utilizador não tenha acesso direto à camada de aplicação ou ao servidor HTTP.

Isto limita a capacidade de configurar:

\begin{itemize}

\item cabeçalhos personalizados (e.g., CSP, HSTS, CORS),

\item regras de resposta dinâmica,

\item e mecanismos de logging ou auditoria específicos [18].

\end{itemize}

\subsubsection{Impacto Prático}

Na Base44, que utiliza Cloudflare como infraestrutura de distribuição, observou-se que os desenvolvedores têm controlo limitado sobre as políticas de segurança no nível HTTP, dependendo inteiramente das definições predefinidas da plataforma.

Essa abordagem reduz a flexibilidade de endurecimento (hardening) da aplicação, comprometendo parcialmente a capacidade de cumprir normas como as do OWASP Secure Headers Project [19].

\subsection{Exposição Acidental de Ficheiros .map (Source Maps)}

\subsubsection{Enquadramento Teórico}

Os source maps (.map) são ficheiros gerados durante o processo de build de aplicações JavaScript ou TypeScript, permitindo mapear o código minificado para a sua versão original.

Embora úteis em contexto de desenvolvimento, estes ficheiros podem expor estruturas internas de código, nomes de variáveis, endpoints e tokens de depuração — representando um risco grave de fuga de informação [20].

A OWASP classifica este tipo de exposição como uma forma de Information Disclosure (A05:2021 - Security Misconfiguration) [21].

\subsubsection{Situação Observada na Plataforma}

Em relatórios de utilizadores, foram mencionadas situações em que os ficheiros .map foram tornados públicos por omissão, devido a políticas de deploy ou configurações de cache da Base44 [22].

No caso da aplicação analisada, os pedidos a ficheiros .map retornaram código 404, o que é positivo; no entanto, as referências sourceMappingURL permaneciam presentes nos scripts minificados, o que representa um risco potencial se a plataforma, num futuro deploy, deixar de bloquear esses ficheiros corretamente.

\subsubsection{Recomendações}

Recomenda-se a remoção das referências sourceMappingURL em builds de produção, e a verificação periódica dos endpoints expostos via ferramentas automatizadas como o Burp Suite ou OWASP ZAP [23].

\subsection{Impossibilidade de Configurar Cabeçalhos de Segurança (CORS, CSP, etc.)}

\subsubsection{Enquadramento Teórico}

Os HTTP Security Headers são mecanismos fundamentais para reforçar a segurança no lado do cliente.

Cabeçalhos como Content-Security-Policy (CSP) e Access-Control-Allow-Origin (CORS) permitem controlar o carregamento de scripts externos, mitigar ataques XSS, e restringir o intercâmbio de dados entre origens [19], [24].

Em plataformas no-code, a ausência de acesso ao servidor impede muitas vezes a definição granular destes cabeçalhos, criando um risco de conformidade parcial com o OWASP Top 10 e outras normas de segurança aplicáveis [25].

\subsubsection{Situação na Base44}

A Base44 não permite configuração manual ou granular de cabeçalhos HTTP de segurança. Isto afeta diretamente a capacidade de mitigar vulnerabilidades como:

\begin{itemize}

\item Cross-Site Scripting (XSS);

\item Data Injection via scripts externos;

\item Resource Manipulation em ambientes de terceiros.

\end{itemize}

Esta limitação obriga o programador a compensar a falta de controlo através de medidas no próprio código (por exemplo, sanitização rigorosa e uso de trusted content sources).

\subsection{Caching Persistente pela CDN}

\subsubsection{Enquadramento Teórico}

O caching distribuído é uma técnica essencial em CDNs para otimizar o desempenho, armazenando recursos em servidores edge.

No entanto, este mecanismo pode criar problemas de propagação lenta após atualizações críticas, especialmente quando ficheiros comprometidos ou sensíveis permanecem em cache por períodos prolongados [26].

Este fenómeno, conhecido como cache inertia, é particularmente perigoso quando uma vulnerabilidade corrigida no servidor demora a refletir-se em toda a rede CDN [27].

\subsubsection{Situação na Base44}

Foi identificado que a Base44, ao operar sobre Cloudflare, aplica políticas agressivas de caching, o que pode atrasar a propagação de correções de segurança ou remoção de ficheiros sensíveis.

Embora benéfico em termos de performance, este comportamento pode atrasar o ciclo de resposta a incidentes (IR), contrariando boas práticas do NIST Computer Security Incident Handling Guide [28].

\subsection{Relação com as Descobertas do Projeto}

As vulnerabilidades observadas — nomeadamente as referências a source maps e a presença de strings HTTP em scripts minificados — alinham-se com as limitações conhecidas da plataforma Base44.

Embora não constituam falhas diretas da aplicação, refletem restrições arquiteturais que limitam a capacidade de aplicar controlos de segurança mais robustos no ambiente de produção.

\subsection{Conclusão}

A compreensão destas limitações estruturais e operacionais da Base44 é essencial para uma análise de segurança completa.

Mesmo quando o código da aplicação segue as boas práticas de segurança, a plataforma de alojamento pode introduzir riscos colaterais — seja por restrições técnicas, políticas de caching ou falta de controlo sobre cabeçalhos HTTP.

Assim, recomenda-se que os programadores:

\begin{enumerate}

    \item Verifiquem regularmente o estado de exposição de ficheiros no build;

    \item Testem a presença de cabeçalhos de segurança via ferramentas automatizadas;

    \item Reavaliem periodicamente a configuração de cache após atualizações críticas.

\end{enumerate}

Estas medidas ajudam a compensar as limitações da plataforma e garantem um nível mais elevado de resiliência e conformidade com o OWASP e o NIST CSF [2], [18].
