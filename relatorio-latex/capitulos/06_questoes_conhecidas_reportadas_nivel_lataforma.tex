\section{Questões Conhecidas e Reportadas ao Nível da Plataforma (Base44)}

Para além das vulnerabilidades específicas detetadas na aplicação em si, é também essencial considerar as limitações estruturais da plataforma de alojamento — neste caso, a Base44.

Tal como documentado em fóruns de utilizadores e fontes técnicas, a Base44 adota uma arquitetura \textit{no-code/low-code} baseada em \textit{deployments edge} através da infraestrutura da Cloudflare, o que traz benefícios de desempenho, mas também restrições de controlo a nível de servidor \cite{ref16, ref2}.

Estas limitações têm implicações diretas no meio de ataque, na gestão de cabeçalhos de segurança, e na rapidez de resposta a incidentes. A seguir detalham-se os principais pontos identificados.

\subsection{Infraestrutura Baseada em Cloudflare e \textit{Deployments Edge}}

\subsubsection{Enquadramento Teórico}

A arquitetura \textit{edge} consiste na distribuição de conteúdos através de múltiplos nós geograficamente dispersos (\textit{edge servers}), com o objetivo de reduzir a latência e aumentar a disponibilidade \cite{ref17}.

Neste modelo, o controlo direto sobre o servidor de origem é frequentemente substituído por políticas de \textit{cache} e regras de \textit{proxy} geridas pela rede CDN (\textit{Content Delivery Network}).

Embora este modelo proporcione resiliência, escalabilidade e mitigação de DDoS, também implica que o utilizador não tenha acesso direto à camada de aplicação ou ao servidor HTTP.

Isto limita a capacidade de configurar:

\begin{itemize}

\item cabeçalhos personalizados (e.g., CSP, HSTS, CORS);

\item regras de resposta dinâmica;

\item e mecanismos de \textit{logging} ou auditoria específicos \cite{ref18}.

\end{itemize}

\subsubsection{Impacto Prático}

Na Base44, que utiliza Cloudflare como infraestrutura de distribuição, observou-se que os programadores têm controlo limitado sobre as políticas de segurança no nível HTTP, dependendo inteiramente das definições predefinidas da plataforma.

Essa abordagem reduz a flexibilidade de endurecimento (\textit{hardening}) da aplicação, comprometendo parcialmente a capacidade de cumprir normas como as do OWASP \textit{Secure Headers Project} \cite{ref19}.

\subsection{Exposição Acidental de Ficheiros .map (\textit{Source Maps})}

\subsubsection{Enquadramento Teórico}

Os \textit{source maps} (.map) são ficheiros gerados durante o processo de \textit{build} de aplicações JavaScript ou TypeScript, permitindo mapear o código minificado para a sua versão original.

Embora úteis em contexto de desenvolvimento, estes ficheiros podem expor estruturas internas de código, nomes de variáveis, \textit{endpoints} e \textit{tokens} de depuração — representando um risco grave de fuga de informação \cite{ref20}.

A OWASP classifica este tipo de exposição como uma forma de \textit{Information Disclosure} (A05:2021 - \textit{Security Misconfiguration}) \cite{ref21}.

\subsubsection{Situação Observada na Plataforma}

Em relatórios de utilizadores, foram mencionadas situações em que os ficheiros .map foram tornados públicos por omissão, devido a políticas de \textit{deploy} ou configurações de \textit{cache} da Base44 \cite{ref22}.

No caso da aplicação analisada, os pedidos a ficheiros .map retornaram código 404, o que é positivo; no entanto, as referências \textit{sourceMappingURL} permaneciam presentes nos scripts minificados, o que representa um risco potencial se a plataforma, num futuro \textit{deploy}, deixar de bloquear esses ficheiros corretamente.

\subsubsection{Recomendações}

Recomenda-se a remoção das referências \textit{sourceMappingURL} em \textit{builds} de produção, e a verificação periódica dos endpoints expostos via ferramentas automatizadas como o Burp Suite ou OWASP ZAP \cite{ref23}.

\subsection{Impossibilidade de Configurar Cabeçalhos de Segurança (CORS, CSP, etc.)}

\subsubsection{Enquadramento Teórico}

Os HTTP \textit{Security Headers} são mecanismos fundamentais para reforçar a segurança no lado do cliente.

Cabeçalhos como \textit{Content-Security-Policy} (CSP) e \textit{Access-Control-Allow-Origin} (CORS) permitem controlar o carregamento de scripts externos, mitigar ataques XSS, e restringir o intercâmbio de dados entre origens \cite{ref19, ref24}.

Em plataformas \textit{no-code}, a ausência de acesso ao servidor impede muitas vezes a definição granular destes cabeçalhos, criando um risco de conformidade parcial com o OWASP Top 10 e outras normas de segurança aplicáveis \cite{ref25}.

\subsubsection{Situação na Base44}

A Base44 não permite configuração manual ou granular de cabeçalhos HTTP de segurança. Isto afeta diretamente a capacidade de mitigar vulnerabilidades como:

\begin{itemize}

\item \textit{Cross-Site Scripting} (XSS);

\item \textit{Data Injection} via \textit{scripts} externos;

\item \textit{Resource Manipulation} em ambientes de terceiros.

\end{itemize}

Esta limitação obriga o programador a compensar a falta de controlo através de medidas no próprio código (por exemplo, sanitização rigorosa e uso de \textit{trusted content sources}).

\subsection{\textit{Caching} Persistente pela CDN}

\subsubsection{Enquadramento Teórico}

O \textit{caching} distribuído é uma técnica essencial em CDNs para otimizar o desempenho, armazenando recursos em servidores \textit{edge}.

No entanto, este mecanismo pode criar problemas de propagação lenta após atualizações críticas, especialmente quando ficheiros comprometidos ou sensíveis permanecem em \textit{cache} por períodos prolongados \cite{ref26}.

Este fenómeno, conhecido como \textit{cache inertia}, é particularmente perigoso quando uma vulnerabilidade corrigida no servidor demora a refletir-se em toda a rede CDN \cite{ref27}.

\subsubsection{Situação na Base44}

Foi identificado que a Base44, ao operar sobre Cloudflare, aplica políticas agressivas de \textit{caching}, o que pode atrasar a propagação de correções de segurança ou remoção de ficheiros sensíveis.

Embora benéfico em termos de performance, este comportamento pode atrasar o ciclo de resposta a incidentes (IR), contrariando boas práticas do NIST \textit{Computer Security Incident Handling Guide} \cite{ref28}.

\subsection{Conclusão}

Embora não constituam falhas diretas da aplicação, estas vulnerabilidades refletem restrições arquiteturais que limitam a capacidade de aplicar controlos de segurança mais robustos no ambiente de produção.

Mesmo quando o código da aplicação segue as boas práticas de segurança, a plataforma de alojamento pode introduzir riscos colaterais — seja por restrições técnicas, políticas de \textit{caching} ou falta de controlo sobre cabeçalhos HTTP.

Assim, recomenda-se que os programadores:

\begin{enumerate}

    \item Verifiquem regularmente o estado de exposição de ficheiros no \textit{build};

    \item Testem a presença de cabeçalhos de segurança via ferramentas automatizadas;

    \item Reavaliem periodicamente a configuração de cache após atualizações críticas.

\end{enumerate}

Estas medidas ajudam a compensar as limitações da plataforma e garantem um nível mais elevado de resiliência e conformidade com o OWASP e o NIST CSF \cite{ref2, ref18}.
