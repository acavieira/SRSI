\section{Conclusão}

O presente trabalho cumpriu o objetivo de analisar e avaliar as vulnerabilidades de segurança no \textit{frontend} de uma aplicação desenvolvida na plataforma \textit{no-code} Base44.
A investigação, conduzida sob uma abordagem de testes de penetração do tipo caixa-cinzenta, confirmou que, mesmo com acesso restrito ao \textit{backend} e à infraestrutura do servidor,
é possível identificar riscos de segurança significativos através da análise do código do lado do cliente (\textit{HTML}, \textit{CSS} e \textit{JavaScript}).

\subsection{Análise dos Resultados e Descobertas Chave}

\paragraph{Validade da Abordagem}
A metodologia de caixa-cinzenta provou ser a mais adequada para este contexto de desenvolvimento \textit{no-code}.
Ao focar-se no código acessível do \textit{frontend}, foi possível aplicar técnicas de análise sobre os recursos disponíveis ao lado do cliente,
superando as limitações impostas pela natureza da plataforma Base44.

\paragraph{Descobertas e Conformidade}
Apesar de não terem sido detetadas vulnerabilidades comuns como CORS Mal Configurado, riscos associados a \textit{Service Workers} ou exposição de \textit{Tokens},
esta verificação demonstrou uma abordagem responsável, alinhada com os princípios do OWASP Top 10 e das normas NIST.
A ausência destas falhas reforça a importância da segregação de segredos \cite{ref12} e do uso de CDNs reputadas.

\paragraph{Limitações Estruturais da Plataforma Base44}
Uma parte crucial da análise revelou que os maiores desafios de segurança advêm das restrições arquiteturais da Base44 \cite{ref16},
que se baseia em \textit{deployments edge} via Cloudflare \cite{ref17}.
Estas restrições impedem a configuração manual e granular de \textit{Security Headers} (como CSP e CORS),
o que limita a capacidade de \textit{hardening} e mitigação de vulnerabilidades como XSS no nível da infraestrutura.

\paragraph{Riscos de Exposição de Informação}
Foi identificada a exposição potencial de ficheiros \textit{.map} (\textit{Source Maps}), que, apesar de terem retornado um código 404,
permanecem referenciados nos \textit{scripts} minificados.
Esta situação alinha-se com relatórios de outros utilizadores da plataforma \cite{ref22} e representa um risco de fuga de informação (OWASP A05:2021) \cite{ref21},
caso a política de bloqueio da plataforma falhe num futuro \textit{deploy}.

\paragraph{Impacto do \textit{Caching} Persistente}
A política de \textit{caching} agressiva da Base44/Cloudflare, embora benéfica para a \textit{performance},
levanta preocupações sobre a lentidão na propagação de correções críticas, o que pode atrasar o ciclo de resposta a incidentes (IR) \cite{ref28}.

\subsection{Recomendações e Trabalho Futuro}

Este estudo sublinha a necessidade de os programadores em ambientes \textit{no-code} assumirem um papel proativo na segurança,
compensando a falta de controlo a nível do servidor com medidas de reforço no código e práticas de auditoria.

Recomenda-se, para futuros projetos na Base44, a adoção das seguintes medidas:

\paragraph{Remoção de Referências a Source Maps}
Eliminar ativamente as referências \textit{sourceMappingURL} em \textit{builds} de produção,
garantindo que mesmo um erro na política de \textit{caching} da plataforma não exponha o código fonte original.

\paragraph{Testes Regulares}
Utilizar ferramentas automatizadas (como \textit{Burp Suite} ou \textit{OWASP ZAP}) \cite{ref23} para verificar a exposição de ficheiros e cabeçalhos de segurança,
integrando a segurança no ciclo de vida do desenvolvimento (\textit{SDLC}) \cite{ref4}.

\paragraph{Gestão de Cache}
Reavaliar periodicamente a configuração de \textit{cache} após a aplicação de correções críticas de segurança,
para garantir a rápida propagação das mitigações implementadas.

Em suma, o projeto contribui para a consciencialização de que o desenvolvimento em plataformas \textit{no-code}, mesmo com IA, não elimina a superfície de ataque.
A segurança da aplicação deve ser vista como uma responsabilidade partilhada,
onde o controlo técnico limitado no ambiente da plataforma exige uma atenção redobrada aos riscos colaterais
e à manutenção de boas práticas no código do lado do cliente \cite{ref51}.