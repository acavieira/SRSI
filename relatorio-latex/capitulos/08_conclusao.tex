\section{Conclusão}

A investigação realizada demonstrou que, apesar da agilidade oferecida por plataformas \textit{no-code} como a Base44, o código gerado automaticamente transporta vulnerabilidades de segurança significativas que podem comprometer a integridade e a confidencialidade dos dados.

Após a identificação e análise das falhas, a implementação de medidas corretivas revelou-se eficaz na maioria dos vetores de ataque. Conforme evidenciado nos resultados, quatro das cinco vulnerabilidades críticas identificadas, incluindo \textit{DOM-based XSS}, \textit{Open Redirects}, \textit{Reverse Tabnabbing} e \textit{Source Map Exposure}, foram mitigadas com sucesso, neutralizando a execução de \textit{scripts} maliciosos e redirecionamentos indevidos.

No entanto, a persistência da vulnerabilidade de \textit{Mixed Content} destaca uma limitação crucial: a impossibilidade de intervenção direta sobre recursos geridos pelo SDK da plataforma (Base44). Este facto corrobora que a automação e a abstração do \textit{no-code} não garantem, \textit{per se}, a conformidade com as boas práticas de segurança, exigindo auditorias manuais contínuas.

Em suma, este projeto contribui para uma compreensão mais profunda dos riscos no desenvolvimento \textit{no-code}, sublinhando a importância de mecanismos de validação em todo o ciclo de vida da aplicação. Reforça-se, assim, a necessidade de uma abordagem de segurança defensiva e proativa, especialmente em cenários onde o programador detém apenas um controlo parcial sobre a infraestrutura subjacente.