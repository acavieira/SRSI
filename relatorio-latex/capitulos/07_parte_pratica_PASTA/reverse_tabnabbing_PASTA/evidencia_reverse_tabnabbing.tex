
\subsubsection{Correção das Ligações Externas Inseguras (Reverse Tabnabbing)}

Para demonstrar a mitigação da vulnerabilidade, foi executada uma Prova de Conceito (PoC) simulando um cenário de ataque, antes e depois da mitigação, onde a página externa força o redirecionamento da aplicação original para um site arbitrário (neste exemplo, \url{https://www.google.com}).

O teste consistiu na injeção do seguinte comando JavaScript na consola da nova aba aberta: \texttt{window.opener.location = 'https://www.google.com';}

\noindent\textbf{1. Cenário Vulnerável (Antes da Correção)}

Conforme ilustrado na Figura \ref{fig:reverse_tabnabbing_antes}, ao executar o comando na consola da aba de destino, a aba de origem (que continha a aplicação Base44) foi imediatamente redirecionada para o Google. Isto comprova que, num cenário real, um atacante poderia substituir a aplicação legítima por uma página de \textit{phishing} idêntica, levando o utilizador a inserir as suas credenciais num site falso.

\begin{figure}[H]
    \centering
    \includegraphics[width=0.8\linewidth]{capitulos/07_parte_pratica_PASTA/reverse_tabnabbing_PASTA/img/reverse_tabnabbing_antes.png}
    \caption{Demonstração da vulnerabilidade de Reverse Tabnabbing. A execução do comando na consola (direita) forçou a aba original (esquerda) a navegar para google.com sem o consentimento do utilizador.}
    \label{fig:reverse_tabnabbing_antes}
\end{figure}

\noindent\textbf{2. Cenário Mitigado (Após a Correção)}

Após a aplicação dos atributos \texttt{rel="noopener noreferrer"}, o mesmo teste foi repetido. Conforme evidenciado na Figura \ref{fig:reverse_tabnabbing_depois}, a tentativa de definir a propriedade \texttt{.location} resultou numa falha, uma vez que o objeto \texttt{window.opener} é agora nulo. A aba original permaneceu segura e inalterada na aplicação Base44.

\begin{figure}[H]
    \centering
    \includegraphics[width=0.8\linewidth]{capitulos/07_parte_pratica_PASTA/reverse_tabnabbing_PASTA/img/reverse_tabnabbing_depois.png}
    \caption{Bloqueio do ataque de Reverse Tabnabbing. A tentativa de redirecionamento gera um erro na consola (TypeError: Cannot set properties of null), confirmando que a aplicação original está protegida contra manipulação externa.}
    \label{fig:reverse_tabnabbing_depois}
\end{figure}

Nota sobre Navegadores Modernos
É importante notar que, embora navegadores modernos baseados em Chromium (v.\,88+) apliquem implicitamente o comportamento \textit{noopener} em links \texttt{target="\_blank"}, a correção explícita implementada é essencial por dois motivos:

\begin{itemize}
    \item Garante a compatibilidade e segurança em navegadores mais antigos (\textit{Legacy Support});
    \item Aplica a diretiva \textit{noreferrer}, que não é assumida por defeito pelos navegadores, protegendo assim a privacidade dos dados de navegação do utilizador.
\end{itemize}
