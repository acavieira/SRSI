
\subsubsection{Correção das Ligações Externas Inseguras (Reverse Tabnabbing)}

No âmbito das correções de segurança implementadas, foi realizada uma procura direcionada aos componentes de \textit{frontend} da aplicação para identificar todas as instâncias de navegação externa insegura. Foram detetados elementos de âncora (\texttt{<a>}) que utilizavam a propriedade \texttt{target="\_blank"} sem as devidas medidas de isolamento de contexto.

Conforme evidenciado nas alterações de código (ver Figura \ref{fig:reverse_tabnabbing_codigo}), a correção foi aplicada nos ficheiros \texttt{src/pages/Layout.jsx} e \texttt{src/pages/Profile.jsx}. Nestes componentes, existiam \textit{links} direcionados para perfis externos (LinkedIn) que expunham a aplicação ao risco de \textit{Reverse Tabnabbing}.

\begin{figure}[hbtp]
    \centering
    \includegraphics[width=0.8\linewidth]{capitulos/07_parte_pratica_PASTA/reverse_tabnabbing_PASTA/img/reverse_tabnabbing_codigo.png}
    \caption{Mitigação de \textit{Reverse Tabnabbing}. \textit{Diff} de código evidenciando a correção de segurança aplicada nos componentes Layout.jsx e Profile.jsx através da adição dos atributos rel.}
    \label{fig:reverse_tabnabbing_codigo}
\end{figure}

A mitigação consistiu na injeção explícita do atributo \texttt{rel="noopener noreferrer"} nas tags afetadas. Esta alteração instrui o navegador a:

\begin{itemize}
    \item \textbf{Noopener}: Garantir que a nova aba é instanciada num processo ou contexto separado, definindo \texttt{window.opener} como nulo.
    \item \textbf{Noreferrer}: Impedir o envio do cabeçalho \textit{Referer}, assegurando que o site de destino não recebe informações sobre a origem do tráfego.
\end{itemize}

Após a recompilação e execução da aplicação, verificou-se que a abertura destes \textit{links} externos mantém a funcionalidade esperada (abertura em nova aba), mas elimina o vetor de ataque anteriormente existente.
