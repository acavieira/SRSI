\subsubsection{Correção da Vulnerabilidade de Open Redirect}

Durante a análise do fluxo de autenticação verificou-se que a aplicação utilizava a expressão:

\begin{verbatim}
User.loginWithRedirect(window.location.href)
\end{verbatim}

A utilização de \texttt{window.location.href} representa um risco, pois o URL completo pode ser manipulado pelo utilizador antes do processo de autenticação. Por exemplo, um atacante pode adicionar parâmetros como:

\begin{verbatim}
/add-today?redirect=https://malicious.com
\end{verbatim}

Caso o código confie no URL completo, existe a possibilidade de provocar um \textit{open redirect}, levando o utilizador a um domínio externo após o login.

Para eliminar este risco, o código foi modificado para usar apenas o caminho interno da aplicação:

\begin{verbatim}
User.loginWithRedirect(window.location.pathname)
\end{verbatim}

As alterações realizadas no código-fonte podem ser visualizadas na Figura \ref{fig:github_open_redirect}.

\begin{figure}[hbtp]
    \centering
    \includegraphics[width=0.95\textwidth]{capitulos/07_parte_pratica_PASTA/open_redirect/img/open_redirect_github.png}
    \caption{Diff do GitHub mostrando a correção do Open Redirect}
    \label{fig:github_open_redirect}
\end{figure}

O \texttt{pathname} contém apenas a rota interna e ignora completamente domínios externos, parâmetros e fragmentos adicionados manualmente. Assim, qualquer tentativa de manipulação externa desaparece no momento da construção do URL de redirecionamento. Desta forma, elimina-se o vetor de ataque e garante-se que o redirecionamento ocorre sempre apenas dentro da própria aplicação.