\subsection{Validação da Correção em Ambiente de Desenvolvimento}

Para confirmar a eficácia da correção, foi utilizado um mock do sistema de autenticação Base44 em ambiente de desenvolvimento. O objetivo foi observar diretamente qual URL a aplicação envia para a função \texttt{loginWithRedirect()} após a alteração implementada.

Primeiro, configurou-se o mock para simular o estado ``não autenticado'' na primeira execução. Isto obriga a aplicação a invocar o redirecionamento, permitindo inspecionar o valor utilizado.

\begin{verbatim}
throw new Error("Mock unauthenticated");
\end{verbatim}

Após gerar o erro intencional, a aplicação chama:

\begin{verbatim}
loginWithRedirect(window.location.pathname)
\end{verbatim}

O mock regista no console o valor recebido e, em seguida, executa um redirecionamento real com esse valor. Isto torna possível confirmar se a aplicação utiliza realmente apenas caminhos internos.

Foram testados dois cenários:

\begin{itemize}
    \item acesso normal à aplicação (por exemplo, \texttt{/add-today});
    \item tentativa de ataque usando parâmetros maliciosos:
    \begin{verbatim}
/add-today?redirect=https://malicious.com
    \end{verbatim}
\end{itemize}

Em ambos os casos, o comportamento observado foi o mesmo: o mock registou sempre apenas o valor interno:

\begin{verbatim}
/add-today
\end{verbatim}

A evidência deste comportamento no console do navegador é apresentada na Figura \ref{fig:devtools_open_redirect}.

\begin{figure}[hbtp]
    \centering
    \includegraphics[width=0.95\textwidth]{capitulos/07_parte_pratica_PASTA/open_redirect/img/open_redirect_devtools.png}
    \caption{Validação da correção via Developer Tools}
    \label{fig:devtools_open_redirect}
\end{figure}

Os parâmetros externos, incluindo URLs maliciosas, não foram utilizados no processo de redirecionamento. Isto confirma que, após a correção, o valor completo do URL deixa de ser relevante e qualquer manipulação feita pelo utilizador é ignorada automaticamente pela lógica da aplicação.

Assim, conclui-se que a vulnerabilidade de \textit{open redirect} foi eliminada com sucesso.