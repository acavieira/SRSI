\subsubsection{Injeções Inseguras no DOM}

Durante a análise foi identificado no ficheiro \texttt{chart.jsx} o uso de 
\texttt{dangerouslySetInnerHTML}, uma função do React que permite a inserção direta 
de código HTML no DOM. Este mecanismo é reconhecido como um \emph{sink} inseguro, 
uma vez que qualquer valor não validado inserido no atributo \texttt{\_\_html} 
pode ser interpretado pelo navegador como código executável, permitindo a ocorrência 
de ataques do tipo DOM-based XSS. A Figura~\ref{fig:codigo-vulneravel-xss} apresenta o 
trecho vulnerável.

\begin{figure}[H]
    \centering
    \includegraphics[width=0.95\textwidth]{capitulos/07_parte_pratica_PASTA/7.2.2_Injecoes_inseguras_no_DOM/img/Figura2.png}
    \caption{Exemplo de código vulnerável XSS}
    \label{fig:codigo-vulneravel-xss}
\end{figure}

Para mitigar esta vulnerabilidade, a geração dinâmica de estilos foi reescrita de forma 
a eliminar completamente o uso de \texttt{dangerouslySetInnerHTML}. Em vez de injetar HTML 
diretamente no DOM, a nova abordagem constrói a folha de estilos como uma \emph{string} 
controlada e atribui-a ao elemento \texttt{<style>} através de \emph{children} seguros do 
React, impedindo a interpretação de conteúdo arbitrário. O código corrigido encontra-se 
representado na Figura~\ref{fig:codigo-corrigido-xss}.

\begin{figure}[H]
    \centering
    \includegraphics[width=0.95\textwidth]{capitulos/07_parte_pratica_PASTA/7.2.2_Injecoes_inseguras_no_DOM/img/Figura3.png}
    \caption{Exemplo do código corrigido}
    \label{fig:codigo-corrigido-xss}
\end{figure}

Para além da remoção do \emph{sink} vulnerável, foram ainda aplicadas validações adicionais,
tais como o uso de \texttt{config || \{\}} para prevenir acessos inesperados a propriedades 
indefinidas e filtragem de valores nulos antes da construção das regras CSS.
