\subsubsection{Reforço da Content-Security-Policy (CSP)}

Durante a fase de implementação das medidas de mitigação, a equipa identificou que a aplicação não possuía qualquer política explícita de \emph{Content-Security-Policy} (CSP), o que aumentava significativamente o risco de execução de código malicioso no navegador, sobretudo no contexto das vulnerabilidades relacionadas com manipulação do DOM descritas em secções anteriores.

A CSP é um dos mecanismos mais eficazes para reduzir a superfície de ataque de XSS no \emph{frontend}, uma vez que impõe restrições rígidas sobre as origens permitidas para scripts, estilos, imagens e outros tipos de conteúdo. Para colmatar esta lacuna, foi introduzida uma \texttt{meta tag} CSP no ficheiro \texttt{index.html} da aplicação, com o objetivo de limitar todas as fontes externas não autorizadas e reforçar a segurança do cliente.

A política final implementada encontra-se abaixo:

\begin{figure}[H]
    \centering
    \includegraphics[width=0.95\textwidth]{capitulos/07_parte_pratica_PASTA/7.2.3_Reforco_da_CSP/img/correcao1.png}
    \caption{Politica final implementada de Content-Security-Policy}
    \label{fig:figuracorrecaoCSP}
\end{figure}

Esta política define regras claras sobre os conteúdos que a aplicação pode carregar. Os scripts e os estilos passam a vir apenas da própria aplicação ou de ligações HTTPS seguras. Os conteúdos incorporados através de \texttt{object-src} deixam de ser permitidos, evitando o uso de ficheiros ou plugins que possam representar um risco. As imagens ficam limitadas a origens consideradas seguras e controladas. Para além disso, todos os pedidos de rede passam a ser feitos através de ligações seguras, reduzindo a probabilidade de interferências externas.
